\chapter{Future work and Improvements}
\setlength{\parskip}{1em}
\section{Future Work}
Instrumentation is one of the major concerns. We need to get sensors that detect if a spot is free or not. Sensors would be connected to the central computer and parking spots availability should be updated at the moment of changing status. 

Algorithm to assign parking spots is ready and working as expected.  Now, we need to make it available to final users. A mobile application should be developed. This application should detect a new user has arrived to campus, run the algorithm, calculate path to be taken and show the path to the parking destination spot. Most of car users at ITESM GDA have an Android or iOS based smart-phone so, application should be developed for both operating systems.  If no smart phone is available, a text message can be sent to user. 

\section{Improvements}
Algorithm improvements can be made in order to make it more effective. For example, we can add user preferences memory; many users will always prefer to park in one specific section no matter what. Changing heuristics values depending on each user can also be another way to add user specific capabilities. In fact, there are many change to change how algorithm behaves; lets first wait to see how it works like it is and then implement these enhancements. 
